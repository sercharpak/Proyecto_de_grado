\documentclass[12pt]{article}

%PACKAGES
\usepackage{graphicx}
\usepackage{epstopdf}
\usepackage[english]{babel}
\usepackage[toc,page]{appendix}
\usepackage{caption}
\usepackage{subcaption}
\usepackage{listings}
\usepackage{subfig}
\usepackage{multicol}

\usepackage{hyperref}
\usepackage{color}
\usepackage[left=3cm,top=3cm,right=3cm,nohead,nofoot]{geometry}
\usepackage{braket}
\usepackage{datenumber}
\usepackage{placeins}
\usepackage{amsmath} % Required for some math elements 
\usepackage{epsfig}

%COMMANDS

\begin{document}


\begin{center}
\begin{figure}
\centering%
\epsfig{file=general/logo-andes.jpg,scale=0.75}%
\end{figure}
\vspace{3 cm}
\FloatBarrier


\begin{center}
\Huge
Installation manual for the segmentation of the aorta artery project (cpPlugins + CreaTools)
\\  
\vspace{1 cm}
\Large Sergio Daniel Hern\'{a}ndez Charpak

\large
%200922618

\vspace{1cm}
\Large
Advisors:\\
Marcela Hernández Hoyos\footnote{Universidad de los Andes, Bogot\'{a}, Colombia}\\
Leonardo Fl\'{o}rez\footnote{Pontificia Universidad Javeriana, Bogot\'{a}, Colombia}\\
Eduardo D\'{a}vila\footnote{Laboratoire CREATIS, Lyon, France}\\


\normalsize
\vspace{1cm}

\today

\vspace{1 cm}
\small 
Universidad de los Andes\\
Facultad de Ingenier\'{i}a, Departamento de Ingenier\'{i}a de Sistemas y Computaci\'{o}n\\
Bogot\'{a}, Colombia\\
\end{center}


\normalsize
\end{center}

\newpage

\section{Installation manual}
%\label{App:install}

The following installation instructions are for a Linux platform operating system. They are
tested for Ubuntu 14.04 from Canonical. Working versions for other operating system platforms
such as Microsoft Windows and Apple MacOS are currently under development as there are still some bugs
to fix. The project uses CMake as compilation helper. It should eventually compile on the three major
box flavors (linux, windows, mac).

\subsection{Libraries: CMake, VTK, ITK, QT}

\begin{itemize}
\item CMake (version $\geq$ 2.8)
\end{itemize}
\begin{par}
CMake is necessary as it is the control of the software compilation process. Our implementation uses
the most recent version at the time, 3.7. The interactive option of CMake, ccmake is recommended and
was used during the current installation. You can get the latest stable version from the
\href{https://cmake.org/cmake/resources/software.html}{\color{blue} Cmake Download Page}
\footnote{\url{https://cmake.org/cmake/resources/software.html}}. A C++ compiler such as \textit{g++}
and \textit{make} are necessary. Once the source code is downloaded and extracted, to install cmake,
run the \textit{bootstrap} script (here use the \textit{--help} option to see the supported options,
such as the \textit{--prefix=custom install directory path} option), then \textit{make} and
\textit{sudo make install} (or \textit{make install} if a custom path was selected and no privileges
are required). An additional library (\textit{libncurses5-dev}) may be necessary for ccmake. \\

In summary:\\

\noindent
\small
\begin{lstlisting}[language=bash]
  1. $ sudo apt-get install libncurses5-dev
  2. $ wget https://cmake.org/files/v3.7/cmake-3.7.1.tar.gz
  3. $ tar -xvf cmake-3.7.1.tar.gz
  4. $ cd cmake-3.7.1/
  5. $ ./bootstrap
  6. $ make
  7. $ sudo make install
  
\end{lstlisting}

\end{par}

\normalsize

\begin{itemize}
\item Qt (version $=$ 4.85)
\end{itemize}
\begin{par}
Although it is possible to download and compile the source code from
\href{http://download.qt.io/archive/ }{\color{blue} Qt download page}
\footnote{\url{http://download.qt.io/archive/ }}, specifically from the \href{http://download.qt.io/archive/qt/4.8/4.8.5/}{\color{blue} Qt 4.85 version folder}
\footnote{\url{http://download.qt.io/archive/qt/4.8/4.8.5/}}, it is preferable to install the
corresponding package through the linux repository running the following command.\\
\noindent
\small
\begin{lstlisting}[language=bash]
  8. $ sudo apt-get install qt4-dev-tools
  
\end{lstlisting}
\end{par}
\normalsize

\begin{itemize}
\item VTK (version $=$ 7.0)
\end{itemize}
\begin{par}
First download the source code from the 
\href{http://www.vtk.org/download/ }{\color{blue} VTK download page}
\footnote{\url{http://www.vtk.org/download/}}, decompress it and create a new folder (where
the code will be built). Move to the new folder and run ccmake from build folder with argument the
source folder. Press \textit{c} to configure and then press \textit{t} to display the advanced
option and be sure that the following options have the values:\\

\noindent
Module\_vtkGUISupportQt: ON\\
Module\_vtkGUISupportQtOpenGL:ON\\
Module\_vtkGUISupportQtSQL: OFF\\
Module\_vtkGUISupportQtWebkit: OFF\\
CMAKE\_BUILD\_TYPE: MinSizeRel\\
BUILD\_SHARED\_LIBS : ON\\
BUILD\_DOCUMENTATION : OFF\\
BUILD\_EXAMPLES : OFF\\
BUILD\_TESTING : OFF\\

Then configure again by pressing \textit{c} and generate the Makefile by pressing \textit{g}.
Then build the code by running \textit{make}  and finally \textit{make install}. VTK will be installed normally in the folder: \textit{/home/(your username)/local/lib/cmake/vtk-7.0}, in my case: \textit{/home/laptop/local/lib/cmake/vtk-7.0}.\\

In summary:\\

\noindent
\small
\begin{lstlisting}[language=bash]
  9. $ wget http://www.vtk.org/files/release/7.0/VTK-7.0.0.tar.gz
 10. $ tar -xvf VTK-7.0.0.tar.gz
 11. $ mkdir VTK_Built
 12. $ cd VTK_Built
 13. $ ccmake ../VTK-7.0.0
 14. Press c 
 15. Press t
 16. Make sure the following options have the correct values:
     Module_vtkGUISupportQt: ON
     Module_vtkGUISupportQtOpenGL:ON
     Module_vtkGUISupportQtSQL: OFF
     Module_vtkGUISupportQtWebkit: OFF
     CMAKE_BUILD_TYPE: MinSizeRel
     BUILD_SHARED_LIBS : ON
     BUILD_DOCUMENTATION : OFF
     BUILD_EXAMPLES : OFF
     BUILD_TESTING : OFF
 17. Press g
 18. $ make
 19. $ make install
  
\end{lstlisting}
\end{par}
\normalsize



\begin{itemize}
\item ITK (version $=$ 4.10)
\end{itemize}
\begin{par}
Download the source code from the 
\href{https://itk.org/ITK/resources/software.html }{\color{blue} ITK download page}
\footnote{\url{https://itk.org/ITK/resources/software.html}}, decompress it and create a new folder (where
the code will be built). Move to the new folder and run ccmake from build folder with argument the
source folder. Press \textit{c} to configure and then press \textit{t} to display the advanced
option and be sure that the following options have the values:\\

\noindent
VTK\_DIR : Folder where VTK is installed. In my case it was: /home/laptop/local/lib/cmake/vtk-7.0\\
Module\_ITKReview : ON\\
Module\_ITKVtkGlue : OFF\\
CMAKE\_BUILD\_TYPE: MinSizeRel\\
BUILD\_SHARED\_LIBS : ON\\
BUILD\_DOCUMENTATION : OFF\\
BUILD\_EXAMPLES : OFF\\
BUILD\_TESTING : OFF\\

Then configure again by pressing \textit{c} and generate the Makefile by pressing \textit{g}.
Then build the code by running \textit{make}  and finally \textit{make install}. ITK will be
installed normally in the folder: \textit{/home/(your username)/local/lib/cmake/ITK-4.10}, in
my case: \textit{/home/laptop/local/lib/cmake/ITK-4.10}.\\


In summary:\\

\noindent
\small
\begin{lstlisting}[language=bash]
 20. $ wget http://downloads.sourceforge.net/project/itk/itk/4.10/
        InsightToolkit-4.10.1.tar.gz?r=https%3A%2F%2Fitk.org%2FITK%2
        Fresources%2Fsoftware.html&ts=1483297536&use_mirror=superb-dca2
 21. $ tar -xvf InsightToolkit-4.10.1.tar.gz
 22. $ mkdir ITK_Built
 23. $ cd ITK_Built
 24. $ ccmake ../InsightToolkit-4.10.1
 25. Press c 
 26. Press t
 27. Make sure the following options have the correct values:
     VTK_DIR : Folder where VTK is installed, for 
     example: /home/laptop/local/lib/cmake/vtk-7.0
     Module_ITKReview : ON
     Module_ITKVtkGlue : OFF
     CMAKE_BUILD_TYPE: MinSizeRel
     BUILD_SHARED_LIBS : ON
     BUILD_DOCUMENTATION : OFF
     BUILD_EXAMPLES : OFF
     BUILD_TESTING : OFF
 28. Press g
 29. $ make
 30. $ make install
  
\end{lstlisting}

\end{par}
\normalsize



\subsection{cpPlugins with plugins: FrontAlgorithms, cpPluginsLeFaucon}

Professor Leonardo Fl\'{o}rez from Pontifica Universidad Javeriana is the main developper for cpPlugins and its plugins. \href{mailito:florez-l@javeriana.edu.co}{\color{blue} Contact Leonardo}
\footnote{\url{florez-l@javeriana.edu.co}} to have access to the source code for cpPlugins, FrontAlgorithms and cpPluginsLeFaucon and in case of having any difficulty related with cpPlugins, FrontAlgorithms and cpPluginsLeFaucon.\\

Once downloaded the source code for cpPlugins, FrontAlgorithms and cpPluginsLeFaucon proceed with the following steps for each project.\\

\begin{itemize}
\item cpPlugins
\end{itemize}
\begin{par}
Decompress the source code and create a new folder (where
the code will be built). Move to the new folder folder and run ccmake from the build folder with argument the
source folder. Press \textit{c} to configure and be sure that the following options have the values. You may have to press c every time an option is configured:\\

\noindent
BUILD\_PipelineEditor : ON\\
BUILD\_EXAMPLES : ON\\
BUILD\_MPRSeeds : ON\\
BUILD\_MPRViewer : ON\\
BUILD\_PipelineEditor : ON\\
BUILD\_SIMPLE\_TESTS : ON\\
CMAKE\_BUILD\_TYPE : Debug\\
CMAKE\_INSTALL\_PREFIX : /usr/local\\
ITK\_DIR : Folder where ITK is installed, for example: /home/laptop/local/lib/cmake/ITK-4.10
QT\_QMAKE\_EXECUTABLE : /usr/bin/qmake\\
USE\_QT4 : ON\\
VTK\_DIR : Folder where VTK is installed, for example: /home/laptop/local/lib/cmake/vtk-7.0\\
cpPlugins\_CONFIG\_NUMBER\_OF\_FIL : 10\\
cpPlugins\_CONFIG\_PROCESS\_DIMEN : 1;2;3\\ 
cpPlugins\_CONFIG\_VISUAL\_DIMENS : 2;3\\
cpPlugins\_Qt4\_VTKWidget : QVTKWidget\\

Then configure again by pressing \textit{c} and generate the Makefile by pressing \textit{g}.
Then build the code by running \textit{make}.\\

In summary:\\

\noindent
\small
\begin{lstlisting}[language=bash]
 31. $ mkdir cpPlugins_Built
 32. $ cd cpPlugins_Built
 33. $ ccmake folder where the source of cpPlugins is.
 34. Press c 
 35. Make sure the following options have the correct values:
     BUILD_PipelineEditor : ON
     BUILD_EXAMPLES : ON
     BUILD_MPRSeeds : ON
     BUILD_MPRViewer : ON
     BUILD_PipelineEditor : ON
     BUILD_SIMPLE_TESTS : ON
     CMAKE_BUILD_TYPE : Debug
     CMAKE_INSTALL_PREFIX : /usr/local
     ITK_DIR : Folder where ITK is installed, for 
      example: /home/laptop/local/lib/cmake/ITK-4.10
     QT_QMAKE_EXECUTABLE : /usr/bin/qmake
     USE_QT4 : ON
     VTK_DIR : Folder where VTK is installed, for 
      example: /home/laptop/local/lib/cmake/vtk-7.0
     cpPlugins_CONFIG_NUMBER_OF_FIL : 10
     cpPlugins_CONFIG_PROCESS_DIMEN : 1;2;3   
     cpPlugins_CONFIG_VISUAL_DIMENS : 2;3
     cpPlugins_Qt4_VTKWidget : QVTKWidget
 36. Press g
 37. $ make
 
\end{lstlisting}

\end{par}
\normalsize

\begin{itemize}
\item FrontAlgorithms
\end{itemize}
\begin{par}

Decompress the source code and create a new folder (where
the code will be built). Move to the new folder folder and run ccmake from the build folder with argument the
source folder. Press \textit{c} to configure and be sure that the following options have the values. You may have to press c every time an option is configured:\\

\noindent
 BUILD\_EXAMPLES : OFF\\
 BUILD\_EXPERIMENTS : OFF\\
 BUILD\_ExperimentationPlugins : OFF\\
 CMAKE\_BUILD\_TYPE : Debug\\
 CMAKE\_INSTALL\_PREFIX : /usr/local\\
 ITK\_DIR : Folder where ITK is installed, for example: /home/laptop/local/lib/cmake/ITK-4.10\\
 USE\_cpPlugins : ON\\
 VTK\_DIR : Folder where VTK is installed, for example: /home/laptop/local/lib/cmake/vtk-7.00\\
 cpPlugins\_BaseLibraries : cpPlugins;cpExtensions;cpPluginsDataObjects;cpBaseQtApplication\\
 cpPlugins\_DIR : Folder where cpPlugins is built.\\

Then configure again by pressing \textit{c} and generate the Makefile by pressing \textit{g}.
Then build the code by running \textit{make}.\\

In summary:\\

\noindent
\small
\begin{lstlisting}[language=bash]
 38. $ mkdir FrontAlgorithms_Built
 39. $ cd FrontAlgorithms_Built
 40. $ ccmake folder where the source of FrontAlgorithms is.
 41. Press c 
 42. Make sure the following options have the correct values:
      BUILD_EXAMPLES : OFF
      BUILD_EXPERIMENTS : OFF
      BUILD_ExperimentationPlugins : OFF
      CMAKE_BUILD_TYPE : Debug
      CMAKE_INSTALL_PREFIX : /usr/local
      ITK_DIR : Folder where ITK is installed, for 
       example: /home/laptop/local/lib/cmake/ITK-4.10
      USE_cpPlugins : ON
      VTK_DIR : Folder where VTK is installed, for 
       example: /home/laptop/local/lib/cmake/vtk-7.00
      cpPlugins_BaseLibraries : cpPlugins;cpExtensions;cpPluginsDataObjects;
       cpBaseQtApplication
      cpPlugins_DIR : Folder where cpPlugins is built.
 43. Press g
 44. $ make
 
\end{lstlisting}


\end{par}
\normalsize

\begin{itemize}
\item cpPluginsLeFaucon
\end{itemize}
\begin{par}

Decompress the source code and create a new folder (where
the code will be built). Move to the new folder folder and run ccmake from the build folder with argument the
source folder. Press \textit{c} to configure and be sure that the following options have the values. You may have to press c every time an option is configured:\\

\noindent
 CMAKE\_BUILD\_TYPE : Debug\\
 CMAKE\_INSTALL\_PREFIX : /usr/local\\
 ITK\_DIR : Folder where ITK is installed, for example: /home/laptop/local/lib/cmake/ITK-4.10\\
 USE\_cpPlugins : ON\\
 VTK\_DIR : Folder where VTK is installed, for example: /home/laptop/local/lib/cmake/vtk-7.00\\
 cpPlugins\_BaseLibraries : cpPlugins;cpExtensions;cpPluginsDataObjects;cpBaseQtApplication\\
 cpPlugins\_DIR : Folder where cpPlugins is built.\\

Then configure again by pressing \textit{c} and generate the Makefile by pressing \textit{g}.
Then build the code by running \textit{make}.\\

In summary:\\

\noindent
\small
\begin{lstlisting}[language=bash]
 38. $ mkdir cpPluginsLeFaucon_Built
 39. $ cd cpPluginsLeFaucon_Built
 40. $ ccmake folder where the source of cpPluginsLeFaucon is.
 41. Press c 
 42. Make sure the following options have the correct values:
      CMAKE_BUILD_TYPE : Debug
      CMAKE_INSTALL_PREFIX : /usr/local
      ITK_DIR : Folder where ITK is installed, for 
       example: /home/laptop/local/lib/cmake/ITK-4.10
      USE_cpPlugins : ON
      VTK_DIR : Folder where VTK is installed, for 
       example: /home/laptop/local/lib/cmake/vtk-7.00
      cpPlugins_BaseLibraries : cpPlugins;cpExtensions;cpPluginsDataObjects;
       cpBaseQtApplication
      cpPlugins_DIR : Folder where cpPlugins is built.
 43. Press g
 44. $ make
 
\end{lstlisting}

\end{par}
\normalsize

\subsection{BBTK and CreaTools}

Researcher Eduardo D\'{a}vila from the CREATIS Laboratory (Lyon, France) is the main developper for BBTK and CreaTools. \href{mailito:Eduardo.Davila@creatis.insa-lyon.fr}{\color{blue} Contact Eduardo}
\footnote{\url{Eduardo.Davila@creatis.insa-lyon.fr}} if there is any difficulty related with BBTK and CreaTools.\\

\begin{itemize}
\item CreaTools
\end{itemize}

Simply download the CreaTools installation script at the \href{https://www.creatis.insa-lyon.fr/site/en/CreatoolsDownload.html}{\color{blue} CreaTools download page}
\footnote{\url{https://www.creatis.insa-lyon.fr/site/en/CreatoolsDownload.html}} and run it with administration privileges. The script will install the libraries necessary for CreaTools and the different tools in the system. \\

In summary:\\

\noindent
\small
\begin{lstlisting}[language=bash]
 45. $ wget http://www.creatis.insa-lyon.fr/software/public/creatools/
        creaTools/Install-Creatools-Bin-Release.sh 
 46. $ source ./Install-Creatools-Bin-Release.sh 
\end{lstlisting}

\begin{itemize}
\item BBTK part from our project
\end{itemize}

Download our project source. In order to do it \href{mailito:sd.hernandez204@uniandes.edu.co}{\color{blue} contact Sergio}\footnote{\url{sd.hernandez204@uniandes.edu.co}}. Decompress it. The project has a folder structure Images, Results, script.\\

In summary:\\

\noindent
\small
\begin{lstlisting}[language=bash]
 47. Download the project source
 48. Decompress the project source
\end{lstlisting}

\subsection{Final adjustments}

In order to connect cpPlugins with our project a few more steps are necessary. First go to the script
folder of our project, then create a link to the executable for cpPlugins with the command 
\textit{ls -s TARGET LINK\_NAME}and finally create a file called \textit{cpPlugins\_PATHS} which will
contain the paths for the built versions of cpPlugins, FrontAlgorithms and cpPluginsLeFaucon.\\

In summary:\\

\noindent
\small
\begin{lstlisting}[language=bash]
 49. $ cd OurProject/script
 50. $ ln -s pathToBuiltCpPlugins/cpPlugins_plugins_ExecutePipeline 
        cpPlugins_plugins_ExcecutePipeline
 51. Create a plain text file called cpPlugins_PATHS with:
     path to our project/script (current path)
     path to the folder of the built version of FrontAlgorithms
     path to the folder of the built version of cpPlugins
     path to the folder of the built version of cpPluginsLeFaucon
\end{lstlisting}

Once all theses steps have been followed, our project should be ready to be executed.



\normalsize
\end{document} 
